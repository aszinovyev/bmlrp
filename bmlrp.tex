\documentclass[conference]{IEEEtran}

%\usepackage[margin=0.6in]{geometry}
%\geometry{letterpaper}

\usepackage[usenames, dvipsnames]{color}
\usepackage{url}
\usepackage{tikz}

\ifCLASSOPTIONcompsoc
    \usepackage[caption=false,font=normalsize,labelfont=sf,textfont=sf]{subfig}
\else
    \usepackage[caption=false,font=footnotesize]{subfig}
\fi

\usepackage{amssymb}

\usepackage{amsthm}
\renewcommand{\qedsymbol}{$\blacksquare$}
\newtheorem{thm}{Theorem}
\theoremstyle{definition}
\newtheorem{defn}{Definition}
\newtheorem{algo}{Algorithm}

\usepackage{algorithm}
\usepackage{algpseudocode}

\usepackage{amsmath}
\DeclareMathOperator*{\argmin}{\arg\!\min}

\newcommand{\urbanxor}{URBAN\_XOR}
\newcommand{\cN}{{\mathcal{N}}}

\title{Binary Multi-Level Routing Protocol for Ad Hoc Networks}

\author{
  Anatoliy Zinovyev and Brian L. Mark \\
  Dept. of  Electrical and Computer Eng. \\
  Volgenau School of Engineering \\
  George Mason University \\
  4400 University Drive, MS 1G5 \\
  Fairfax, VA 22030 \\
  email:   \{  azinovye, bmark \}@gmu.edu \\
}


\begin{document}

\maketitle

\begin{abstract}
Routing in mobile ad hoc networks (MANETs) presents a big challenge, especially when support for a large number of nodes is needed. This paper extends the local visibility concept of a DHT-based \urbanxor\ routing protocol \cite{urbanxor}, aiming to reduce routing table size while keeping efficiency high. Our main contribution is providing a guarantee that if any two nodes are connected through other nodes, they are able to communicate with each other. We propose a new route acquisition method that aims to reduce the total amount of overhead traffic and handle mobility well. In addition, we introduce an abstraction for describing the network structure that makes it easy to understand and analyze. Finally, we make our first attempt to simulate the protocol on a large number of nodes to support our claims. Compared to existing approaches in ad hoc routing, the new protocol supports the following features: guaranteed delivery, scalability, absence of single points of failure, low path-stretch, support of multi-path connectivity, mobility.
\end{abstract}

\begin{figure*}[!t]
    \centering
    
    \newcommand{\nodescl}{0.7}
    \newcommand{\edgescl}{1.3}
    \newcommand{\wn}{\node[circle, draw, very thin, scale=\nodescl]}
\newcommand{\bn}{\node[circle, draw, very thin, scale=\nodescl, fill=black!25]}
\newcommand{\Wn}{\wn[ultra thick]}
\newcommand{\Bn}{\bn[ultra thick]}

\newcommand{\coordscl}[1] {#1*\edgescl}


    \subfloat[Level 0] {
        \begin{tikzpicture}
            \Bn[draw=blue] (n0) at (0,0) {11100};
            \Wn (n1) at (\coordscl{1}, \coordscl{0}) {01100};
            \Bn (n2) at (\coordscl{2}, \coordscl{0}) {10010};
            \bn (n3) at (\coordscl{3}, \coordscl{0}) {11001};
            \Bn (n4) at (\coordscl{0}, \coordscl{1}) {10001};
            \Bn (n5) at (\coordscl{1}, \coordscl{1}) {11111};
            \Wn (n6) at (\coordscl{2}, \coordscl{1}) {01010};
            \wn (n7) at (\coordscl{3}, \coordscl{1}) {00110};
            \Bn (n8) at (\coordscl{0}, \coordscl{2}) {10110};
            \Wn (n9) at (\coordscl{1}, \coordscl{2}) {01011};
            \bn (n10) at (\coordscl{2}, \coordscl{2}) {11010};
            \wn (n11) at (\coordscl{3}, \coordscl{2}) {01001};
            \bn (n12) at (\coordscl{0}, \coordscl{3}) {11011};
            \bn (n13) at (\coordscl{1}, \coordscl{3}) {11000};
            \wn (n14) at (\coordscl{2}, \coordscl{3}) {00111};
            \wn[draw=red] (n15) at (\coordscl{3}, \coordscl{3}) {01000};

            \draw[->,ultra thick]
                (n0) -- (n1);

            \draw[ultra thick]
                (n1) -- (n2)
                (n4) -- (n5)
                (n5) -- (n6)
                (n8) -- (n9)
                (n0) -- (n4)
                (n1) -- (n5)
                (n4) -- (n8)
                (n5) -- (n9)
                ;

            \draw
                (n2) -- (n3)
                (n6) -- (n7)
                (n9) -- (n10)
                (n10) -- (n11)
                (n12) -- (n13)
                (n13) -- (n14)
                (n14) -- (n15)
                (n2) -- (n6)
                (n3) -- (n7)
                (n6) -- (n10)
                (n7) -- (n11)
                (n8) -- (n12)
                (n9) -- (n13)
                (n10) -- (n14)
                (n11) -- (n15)
                ;

        \end{tikzpicture}

        \label{fig:bmlrp-routing-l0}
    }%
    %
    \hfil%
    %
    \subfloat[Level 1] {
        \begin{tikzpicture}
            \Bn[draw=blue] (n1) at (\coordscl{1}, \coordscl{0}) {01100};
            \Bn (n6) at (\coordscl{2}, \coordscl{1}) {01010};
            \Wn (n7) at (\coordscl{3}, \coordscl{1}) {00110};
            \Bn (n9) at (\coordscl{1}, \coordscl{2}) {01011};
            \Bn (n11) at (\coordscl{3}, \coordscl{2}) {01001};
            \Wn (n14) at (\coordscl{2}, \coordscl{3}) {00111};
            \bn[draw=red] (n15) at (\coordscl{3}, \coordscl{3}) {01000};

            \draw[ultra thick]
                (n1) -- (n6)
                (n1) -- (n7)
                (n1) -- (n9)
                (n6) -- (n7)
                (n6) -- (n9)
                (n6) -- (n11)
                (n6) -- (n14)
                (n7) -- (n11)
                (n9) -- (n11)
                (n9) -- (n14)
                (n9) -- (n11)
                (n11) -- (n14)
                ;

            \draw
                (n11) -- (n15)
                (n14) -- (n15)
                ;
        \end{tikzpicture}

        \label{fig:bmlrp-routing-l1}
    }%
    %
    \hfil%
    %
    \subfloat[Level 2] {
        \begin{tikzpicture}
            \Bn[draw=blue] (n1) at (\coordscl{2}, \coordscl{0}) {01100};
            \Wn (n6) at (\coordscl{2}, \coordscl{1}) {01010};
            \Wn (n9) at (\coordscl{1}, \coordscl{2}) {01011};
            \Wn (n11) at (\coordscl{3}, \coordscl{2}) {01001};
            \wn[draw=red] (n15) at (\coordscl{2.5}, \coordscl{3}) {01000};

            \draw[->,ultra thick]
                (n1) -- (n6);

            \draw[ultra thick]
                (n1) -- (n9)
                (n1) -- (n11)
                ;

            \draw
                (n6) -- (n9)
                (n6) -- (n11)
                (n6) -- (n15)
                (n9) -- (n11)
                (n9) -- (n15)
                (n11) -- (n15)
                ;
        \end{tikzpicture}

        \label{fig:bmlrp-routing-l2}
    }%
    %
    \hfil%
    %
    \subfloat[Level 3] {
        \begin{tikzpicture}
            \Bn[draw=blue] (n6) at (\coordscl{0.75}, \coordscl{0}) {01010};
            \Bn (n9) at (\coordscl{0}, \coordscl{0.75}) {01011};
            \Wn (n11) at (\coordscl{1.5}, \coordscl{0.75}) {01001};
            \Wn[draw=red] (n15) at (\coordscl{0.75}, \coordscl{1.5}) {01000};

            \draw[->,ultra thick]
                (n6) -- (n15);

            \draw[ultra thick]
                (n6) -- (n9)
                (n6) -- (n11)
                (n6) -- (n15)
                (n9) -- (n11)
                (n9) -- (n15)
                ;

            \draw
                (n11) -- (n15);

        \end{tikzpicture}

        \label{fig:bmlrp-routing-l3}
    }
    
    \caption{Routing in BMLRP}
    \label{fig:bmlrp-routing}
\end{figure*}


\section{Introduction}
\label{sec:introduction}

DHT-based routing protocols in MANETs are interesting for number of reasons. First is their scalability properties achieved by creating a special network structure and reducing the routing table size. Second is their simplicity relative to clustering-based approaches. In clustering-based protocols, nodes are divided into groups, which are subsequently split into smaller groups~\cite{Hong}. Although, such an approach can be scalable to larger networks, it can be difficult to implement. In addition, clustering-based protocols often use critical nodes for controlling clusters \cite{Hong} and dynamic addresses \cite{Hong, dart, ntk}, which may compromise the stability of the network.

DHT-based routing, however, often suffers from the mismatch problem resulting in a high path-stretch\footnote{The path-stretch is defined as $ \frac{{\rm len}(a,b)}{{\rm dist}(a,b)} $, where ${\rm len}(a,b)$ is the route length between nodes $a$ and $b$ found by the routing protocol and ${\rm dist}(a,b)$ is length of the shortest path between $a$ and $b$. The length of a path can be measured as either the latency or the number of hops}. In such networks, every node has its own logical identifier (LID) and also stores information about a portion of other nodes with certain LIDs, such that any node should be able to send information to any other node. Because in such networks routing is done on top of the logical structure and the logical addresses do not necessarily resemble the physical location, the number of physical hops through which data is traversed is often far from being optimal \cite{Abid}.

A number of interesting solutions have been proposed for minimizing the path-stretch. For example, the Virtual Ring Routing protocol builds a virtual ring where nodes are ordered according to their LIDs~\cite{VRR}. Each node maintains a record of constant number of nodes with closest addresses and routes to them. When routing data, the next hop with the closest address to the destination's identifier is chosen. The guarantee of constant path-stretch relies on the fact that each node knows $O(\sqrt{N})$ nodes, and thus the probability that a node will know the route to the destination is $O(\frac{1}{\sqrt{N}})$. Therefore, the expected number of traversed nodes is $O(\sqrt{N})$~\cite{VRR}. Another approach is taken by the 3D~routing protocol, which ensures good path-stretch properties by embedding the node LIDs into a 3-dimensional space~\cite{3drp}. Thus, forwarding data is as simple as sending it in the ``right direction.'' Both approaches, however, fail to deal with the network mobility and merging/splitting operations. 

The protocol proposed in this paper is closest in approach to the recent
\urbanxor\ protocol~\cite{Pasquini}, which employs a Kademlia DHT~\cite{kademlia} inspired approach for building the network structure. Each node maintains $n$ buckets for storing information about other nodes with address prefixes equal to the node's address prefix. When routing information to any node, the next hop with the longest matching prefix is chosen from among the $n$~buckets. Path efficiency is achieved by employing the concept of local visibility which prioritizes physically close nodes in the routing table. By maintaining this network structure, mobility support turns out to be better than in other protocols.

The \urbanxor\ protocol, however, has several drawbacks. The most prominent one is the absence of a guarantee that if two nodes are indirectly connected they are able to communicate \cite{Pasquini}. Another objection is the method of acquiring paths to other nodes, which tends to generate much overhead traffic in mobile scenarios. The protocol proposed in this paper aims to solve both problems altogether. {\color{orange} In addition, it decreases the path-stretch even further by incorporating a novel algorithm that selects near-optimal routes without maintaining them.}

The rest of the paper is organized as follows. Section~\ref{sec:urbanxor} describes the \urbanxor\ routing protocol in more detail. Section~\ref{sec:bmlrp} introduces the new protocol, which overcomes the drawbacks of \urbanxor. 
Section~\ref{sec:analysis} discusses routing properties of the new approach. 
Section~\ref{sec:conclusion} concludes the paper.


\section{\urbanxor}
\label{sec:urbanxor}

In the \urbanxor\ protocol each node has a unique permanent $n$-bit identifier, that is randomly generated before connecting to the network. The routing table of every node is organized into $n$ buckets, each of size $K$. We will denote $l(a,b)$ as a function of two addresses. The value of $l(a,b)$ is the length of the longest common prefix of $a$ and $b$. For instance, $l(01100, 01000) = 2$ because the biggest common prefix is $01$. When a node $a$ discovers a new neighbor $b$, identifier $b$ is added to the bucket number $l(a,b)$ of node $a$.

After the node connects to its direct (both physical and virtual) neighbors, it starts filling the $n$ buckets by asking for missing nodes. When a new node is discovered, an abstract virtual link is built to it. Thus, multi-hop routes in the physical space are stored in a reduced form. At the same time nodes passively overhear the traffic to find new routes without loading the network. The local visibility concept ensures that only physically close nodes are stored in the routing table.

Similarly to Kademlia DHT \cite{kademlia}, when node $a$ is routing data to $b$, the first node selects an address $c$ from its routing table such that $l(b,c)$ is maximized. Because each next node that forwards the data has longer common prefix with the destination, the data eventually arrives. For this to be always true, each node in the network must have at least one entry in each bucket if an appropriate identifier exists. The \urbanxor\ protocol, however, might maintain empty buckets and fail to satisfy the property \cite{Pasquini}. When two physical neighbors disconnect, all virtual routes passing through the destroyed link must be eliminated \cite{Pasquini}. This may result in considerable traffic overhead generated when the node is moving or has an unstable link to a physical neighbor.


\section{The Proposed Protocol}
\label{sec:bmlrp}

Consider a node with a unique permanent $n$-bit address $a$, similarly to \urbanxor. We say that all nodes $b$ in the network form a level-$i$ network with respect to $a$ if $l(a,b) \ge i$. For instance, the level-0 network will contain all nodes, which is also defined as \emph{physical network}. Level-3 network will contain all nodes that have the first 3 bits equal to the first 3 bits of $a$. Because we assume all identifiers are distributed evenly, the number of level-3 nodes will be approximately 8 times smaller than the physical network size. As a general rule, the level-$i$ network contains approximately $\frac{N}{2^i}$ nodes, where $N$ is the size of the physical network.

After a node joins the physical network, it starts creating virtual links to the nearby level-1 nodes. In a sense, it connects to them the same way it connects to the physical neighbors. We now abstractly consider level-1 network similar to level-0 network by all properties. The idea is to ascend up to the point where the number of close level-$m$ neighbors is equal to the total number of all level-$m$ nodes. When node $a$ is routing data to node $d$, it will pick the next hop $b$ from the level-$l(a,d)$ network of $a$ such that $l(b,d) > l(a,d)$. In other words, the first bit that is not equal between $a$ and $d$ must be equal between $b$ and $d$. Similarly to Kademlia DHT and \urbanxor, each new node routing traffic has a longer common prefix with $d$.

\begin{figure*}[!t]
    \centering

    \newcommand{\nodescl}{1}
    \newcommand{\edgescl}{1}
    \newcommand{\wn}{\node[circle, draw, very thin, scale=\nodescl]}
\newcommand{\bn}{\node[circle, draw, very thin, scale=\nodescl, fill=black!25]}
\newcommand{\Wn}{\wn[ultra thick]}
\newcommand{\Bn}{\bn[ultra thick]}

\newcommand{\coordscl}[1] {#1*\edgescl}


    \subfloat[Level-$i$ network] {
        \begin{tikzpicture}
            \wn[draw=blue] (n0) at (\coordscl{0}, \coordscl{0}) {0};
            \wn (n1) at (\coordscl{1}, \coordscl{0}) {1};
            \wn (n2) at (\coordscl{2}, \coordscl{0}) {2};
            \wn (n3) at (\coordscl{3}, \coordscl{0}) {3};
            \bn (n4) at (\coordscl{0}, \coordscl{1}) {4};
            \wn (n5) at (\coordscl{1}, \coordscl{1}) {5};
            \bn (n6) at (\coordscl{2}, \coordscl{1}) {6};
            \bn (n7) at (\coordscl{3}, \coordscl{1}) {7};
            \wn (n8) at (\coordscl{0}, \coordscl{2}) {8};
            \wn (n9) at (\coordscl{1}, \coordscl{2}) {9};
            \wn (n10) at (\coordscl{2}, \coordscl{2}) {10};
            \wn (n11) at (\coordscl{3}, \coordscl{2}) {11};

            \draw[dashed]
                (n0) -- (n1)
                (n0) -- (n4)
                ;

            \draw
                (n1) -- (n2)
                (n2) -- (n3)
                (n2) -- (n6)
                (n6) -- (n5)
                (n7) -- (n6)
                (n4) -- (n8)
                (n11) -- (n7)
                (n8) -- (n9)
                (n9) -- (n10)
                (n10) -- (n11)
                (n1) -- (n5)
                (n3) -- (n7)
                (n4) -- (n5)
                ;
        \end{tikzpicture}
        
        \label{fig:bmlrp-routes-all}
    }%
    %
    \hfil%
    %
    \subfloat[Graph $G_0$ of node 1] {
        \begin{tikzpicture}
            \wn[draw=blue] (n0) at (\coordscl{0}, \coordscl{0}) {0};
            \Wn[draw=green] (n1) at (\coordscl{1}, \coordscl{0}) {1};
            \Wn (n2) at (\coordscl{2}, \coordscl{0}) {2};
            \wn (n3) at (\coordscl{3}, \coordscl{0}) {3};
            \Bn (n4) at (\coordscl{0}, \coordscl{1}) {4};
            \Wn (n5) at (\coordscl{1}, \coordscl{1}) {5};
            \Bn (n6) at (\coordscl{2}, \coordscl{1}) {6};
            \bn (n7) at (\coordscl{3}, \coordscl{1}) {7};
            
            \draw[dashed]
                (n0) -- (n1);

            \draw[ultra thick]
                (n1) -- (n2)
                (n4) -- (n5)
                (n1) -- (n5)
                (n2) -- (n6)
                (n5) -- (n6)
                ;

            \draw
                (n2) -- (n3)
                (n3) -- (n7)
                ;                
        \end{tikzpicture}

        \label{fig:bmlrp-routes-nb1}
    }%
    %
    \hfil%
    %
    \subfloat[Graph $G_0$ of node 4] {
        \begin{tikzpicture}
            \wn[draw=blue] (n0) at (\coordscl{0}, \coordscl{0}) {0};
            \wn (n1) at (\coordscl{1}, \coordscl{0}) {1};
            \wn (n2) at (\coordscl{2}, \coordscl{0}) {2};
            \Bn[draw=green] (n4) at (\coordscl{0}, \coordscl{1}) {4};
            \Wn (n5) at (\coordscl{1}, \coordscl{1}) {5};
            \bn (n6) at (\coordscl{2}, \coordscl{1}) {6};
            \bn (n7) at (\coordscl{3}, \coordscl{1}) {7};
            \Wn (n8) at (\coordscl{0}, \coordscl{2}) {8};
            \wn (n9) at (\coordscl{1}, \coordscl{2}) {9};
            \wn (n10) at (\coordscl{2}, \coordscl{2}) {10};
            \wn (n11) at (\coordscl{3}, \coordscl{2}) {11};

            \draw[dashed]
                (n0) -- (n4);

            \draw[ultra thick]
                (n4) -- (n5)
                (n4) -- (n8)
                ;

            \draw
                (n1) -- (n2)
                (n1) -- (n5)
                (n2) -- (n6)
                (n5) -- (n6)
                (n7) -- (n11)
                (n8) -- (n9)
                (n9) -- (n10)
                (n10) -- (n11)
                ;
        \end{tikzpicture}

        \label{fig:bmlrp-routes-nb2}
    }%
    %
    \hfil%
    %
    \subfloat[Graph $G$ of node 0] {
        \begin{tikzpicture}
            \wn[draw=blue] (n0) at (\coordscl{0}, \coordscl{0}) {0};
            \wn (n1) at (\coordscl{1}, \coordscl{0}) {1};
            \wn (n2) at (\coordscl{2}, \coordscl{0}) {2};
            \bn (n4) at (\coordscl{0}, \coordscl{1}) {4};
            \wn (n5) at (\coordscl{1}, \coordscl{1}) {5};
            \bn (n6) at (\coordscl{2}, \coordscl{1}) {6};
            \wn (n8) at (\coordscl{0}, \coordscl{2}) {8};

            \draw
                (n0) -- (n1)
                (n1) -- (n2)
                (n0) -- (n4)
                (n1) -- (n5)
                (n2) -- (n6)
                (n4) -- (n5)
                (n5) -- (n6)
                (n4) -- (n8)
                ;
        \end{tikzpicture}

        \label{fig:bmlrp-routes-visible}
    }

    \caption{Routes acquisition}
    \label{fig:bmlrp-routes}
\end{figure*}


For simplicity, from now on we will label a level-$i$ node white if the $(i+1)$-th bit of its address is 0; otherwise the node is black. Figure~\ref{fig:bmlrp-routing} shows a simplified example of routing data from 11100 to 01000, with subfigures demonstrating different levels of the network with respect to the destination. The blue nodes denote the routing nodes; the red node on each level indicates the destination. Bold entries indicate nodes and links known by the routing node. The journey starts on the physical level depicted by Figure~\ref{fig:bmlrp-routing-l0}. The source and destination nodes differ in the first bit, therefore 11100 transmits the data to a close physical white node -- 01100. Figure~\ref{fig:bmlrp-routing-l1} shows Level-1 network with respect to the destination, where the routing node is now 01100. Because it is equal in color to the destination, 01100 will forward the data through one of its level-2 neighbors. When node 01010 receives the message, it will find the destination address in its routing table level~3 and will route the data through an intermediate level-3 node.

We will now examine how nodes discover other close level-$i$ nodes.

\begin{defn}
    We say that nodes $a$ and $b$ are connected on level~$i$ if $a$ and $b$ send information about close level-$i$ nodes to each other according to Algorithm~\ref{alg:routes}.
    
    \label{defn:connected}
\end{defn}

\begin{algorithm}[H]
    \caption{Find routes}
    
    \begin{itemize} 

        \item   For each connected neighbor $b$ node $a$ constructs a graph $G_b$ by merging the graphs received from all of its neighbors except $b$ and adding all direct edges to these neighbors, and adding the edge $a$--$b$. Node $a$ then calculates a subset $G_{outb}$ of $G_b$ defined as follows.

        \item   Consider every graph chain $b$--$a$--$c_1$--$c_2$--...--$c_k$--$d$ in $G_b$, where $b$, $a$, $c_i$ $(1 \le i \le k)$, $d$ are different; call this graph chain $gc$. Without loss of generality, let $a$ be white. All nodes and vertices of $gc$ will be included in $G_{outb}$ if: 1) the nodes $c_i$ are white, $d$ is black, and 2) for any $c_i$ there is no black node $f$ in $G_b$, such that the distance from $f$ to $c_i$ is smaller than the distance from $c_i$ to $d$ in $gc$ and the distance from $c_i$ to $b$ in $gc$.

        \item   Exclude node $b$ from $G_{outb}$.

        \item   If node $b$ has just connected to $a$, $a$ will calculate and send out $G_{outb}$. Later, for any change in $G_b$, $a$ will recalculate $G_{outb}$ and send out the difference, such that $b$ can construct the new $G_{outb}$ without incurring much traffic.

        \item   Each node combines all graphs received from its neighbors and uses the new graph $G$ for routing.
    \end{itemize}

    \label{alg:routes}
\end{algorithm}

In Figure~\ref{fig:bmlrp-routes} node 0 connects to nodes 1 and 4. By Algorithm~\ref{alg:routes}, node's 1 graph $G_0$ is depicted in Figure~\ref{fig:bmlrp-routes-nb1}. Chains that will be included in $G_{out0}$ are 0--1--5--4, 0--1--5--6, 0--1--2--6. Chain 0--1--2--3--7 will not be included because the distance from 6 to 2 is shorter than both chains 0--1--2 and 2--3--7. Figure~\ref{fig:bmlrp-routes-nb2} shows a similar graph of node 4. Here, the only 2 chains satisfying the requirement are 0--4--5 and 0--4--8. Node 0 combines the information received from the nodes 1,4 and forms a routing table $G$ depicted in Figure~\ref{fig:bmlrp-routes-visible}.

All updates of links communicated between neighbors are sent in a proper order. The strategy is to proactively maintain only up-to-date routes. This can be explained as follows. Suppose, node $a$ knows about link $b$--$c_1$ that breaks. By the Algorithm~\ref{alg:routes}, for every graph chain $b$--$c_1$--$c_2$--...--$c_k$--$a$ stored in memory of $a$, $c_k$--$a$ and every connection $c_j$--$c_{j+1}$ must be alive. Therefore, if nodes $b$ and $c_1$ disconnect, the information will be propagated through every such chain and record of the link will be erased.

To ensure connectivity between any two nodes in the network, it is sufficient to make every same-color node in level-$i$ graph $G$ a neighbor on level~$(i+1)$. It can be foreseen, however, that this will lead to big routing tables on higher levels. Therefore, a more restrictive approach is needed. Algorithm~\ref{alg:connect} characterizes which nodes are necessary to connect to in order to satisfy the connectivity requirement. Node $a$ indicates the current node. The graph $G$ is assumed to be a level-$i$ routing table of $a$. $\cN(x)$ denotes the set of all neighbors of $x$ in $G$. The procedure  connect$(x)$ indicates that $x$ will become a level-$(i+1)$ neighbor of $a$. Function bit$(node, l)$ returns the $(i + l)$-th bit of the $node$'s address, assuming the first bit is number 0.

\begin{defn}
    Let \emph{neighborhood} of node $x$ be the set of all neighbors of $x$ of the opposite color.
    
    \label{defn:neighborhood}
\end{defn}

\begin{algorithm}
    \begin{algorithmic}[1]
        \caption{Connect node $a$ to necessary nodes}
        \label{alg:connect}

        \Procedure{ConnectToNecessaryNodes}{}
            \For {$ x \in \cN(a) \setminus {\rm NH}(a) $}
                \State connect($x$)
            \EndFor

            \For {$ x \in {\rm NH}(a) $}
                \State ConnectInside(${\rm NH}(x), 0$)

                \For {$ y \in \cN(x) \setminus {\rm NH}(x) $}
                    \State $Used \gets \{false\}$
                    \State $Used[x] \gets true$
                    \State ConnectNeighborhoods($x, y$)
                \EndFor
            \EndFor
        \EndProcedure
        
        \algstore{connect}
    \end{algorithmic}
\end{algorithm}

\begin{algorithm}
    \begin{algorithmic}[1]
        \algrestore{connect}

        \Function{NH}{$node$}   \Comment{returns neighborhood of $node$}
            \State \textbf{return} $ \{x \in \cN(node): bit(x, 0) \ne bit(node, 0)\} $
        \EndFunction

        \algstore{connect}
    \end{algorithmic}
\end{algorithm}

\begin{algorithm}
    \begin{algorithmic}[1]
        \algrestore{connect}

        \Procedure{ConnectInside}{$nodes, l$}
            \State $white \gets \{x \in nodes: bit(x, l) = 0\}$
            \State $black \gets \{x \in nodes: bit(x, l) = 1\}$
            
            \If {$ white = \emptyset$ \textbf{or} $black = \emptyset$}
                \State \textbf{return}
            \EndIf

            \State $w,b \gets \argmin_{x \in white, y \in black}{\rm XOR}(x, y)$

            \If {$bit(a, l) = 0$}
                \If {$w = a$}
                    \State connect($b$)
                \EndIf
                
                \State ConnectInside($white$, $l + 1$)
            \Else
                \If {$b = a$}
                    \State connect($w$)
                \EndIf

                \State ConnectInside($black$, $l + 1$)
            \EndIf
        \EndProcedure
        
        \algstore{connect}
    \end{algorithmic}
\end{algorithm}

\begin{algorithm}
    \begin{algorithmic}[1]
        \algrestore{connect}
        
        \Procedure{ConnectNeighborhoods}{$from, to$}
            \If {$Used[to]$} 
                \State \textbf{return}      %\Comment {we were here before}
            \EndIf
                
            \State $Used[to] \gets true$

            \If {$ {\rm NH(to)} \ne \emptyset $}
                \State $f, t \gets \argmin_{x \in {\rm NH}(from), y \in {\rm NH}(to)}{\rm XOR}(x, y)$

                \If {$f = a$}
                    \State connect($t$)
                \EndIf
            \Else
                \For {$next \in \cN(to) \setminus {\rm NH}(to)$}
                    \State ConnectNeighborhoods($from, next$)
                \EndFor
            \EndIf

        \EndProcedure
    \end{algorithmic}
\end{algorithm}

The first thing Algorithm~\ref{alg:connect} does is connecting to all same-colored level-$i$ neighbors, so that they also become level-$(i+1)$ neighbors. Second, for every neighbor $b$ of different color it connects to some nodes in the neighborhood of $b$, such that a connected tree of nodes in the neighborhood of $b$ is formed. Finally, the algorithm recursively connects neighborhood of $b$ to other local neighborhoods.

Figure~\ref{fig:bmlrp-connect} illustrates this process (solid edges show level-$i$ links, dashed edges indicate level-$(i+1)$ connections). First, note how black neighbors of white nodes are organized into trees. Additionally, some black nodes are chosen to connect to black nodes of other close neighborhoods. Node 4 calls ConnectNeighborhoods$(0, 5)$ and stops the recursion. On the other hand, node 2 does not find any black neighbors of node 11 and calls ConnectNeighborhoods$(0, 12)$ from ConnectNeighborhoods$(0, 11)$.

\begin{figure*}[!t]
    \centering

    \newcommand{\nodescl}{1}
    \newcommand{\edgescl}{1}
    \newcommand{\wn}{\node[circle, draw, very thin, scale=\nodescl]}
\newcommand{\bn}{\node[circle, draw, very thin, scale=\nodescl, fill=black!25]}
\newcommand{\Wn}{\wn[ultra thick]}
\newcommand{\Bn}{\bn[ultra thick]}

\newcommand{\coordscl}[1] {#1*\edgescl}


    \begin{tikzpicture}
        \wn (n0) at (\coordscl{0}, \coordscl{0}) {0};
        \bn (n1) at (\coordscl{1}, \coordscl{1}) {1};
        \bn (n2) at (\coordscl{1}, \coordscl{-1}) {2};
        \bn (n3) at (\coordscl{-1}, \coordscl{-1}) {3};
        \bn (n4) at (\coordscl{-1}, \coordscl{1}) {4};
        \wn (n5) at (\coordscl{-2}, \coordscl{0}) {5};
        \bn (n6) at (\coordscl{-3}, \coordscl{1}) {6};
        \bn (n7) at (\coordscl{-3}, \coordscl{-1}) {7};
        \wn (n8) at (\coordscl{-4}, \coordscl{0}) {8};
        \bn (n9) at (\coordscl{-5}, \coordscl{1}) {9};
        \bn (n10) at (\coordscl{-5}, \coordscl{-1}) {10};
        \wn (n11) at (\coordscl{2}, \coordscl{0}) {11};
        \wn (n12) at (\coordscl{4}, \coordscl{0}) {12};
        \bn (n13) at (\coordscl{5}, \coordscl{1}) {13};
        \bn (n14) at (\coordscl{5}, \coordscl{-1}) {14};
        \bn (n15) at (\coordscl{6}, \coordscl{0}) {14};

        \draw[dashed]
            (n1) -- (n2)
            (n2) -- (n3)
            (n3) -- (n4)
            (n4) -- (n1)
            (n4) -- (n6)
            (n6) -- (n7)
            (n7) -- (n10)
            (n9) -- (n10)
            (n2) -- (n14)
            (n13) -- (n14)
            (n14) -- (n15)
            (n15) -- (n13)
            ;

        \draw
            (n0) -- (n1)
            (n0) -- (n2)
            (n0) -- (n3)
            (n0) -- (n4)
            (n0) -- (n5)
            (n5) -- (n6)
            (n5) -- (n7)
            (n5) -- (n8)
            (n8) -- (n9)
            (n8) -- (n10)
            (n0) -- (n1)
            (n0) -- (n2)
            (n0) -- (n11)
            (n11) -- (n12)
            (n12) -- (n13)
            (n12) -- (n14)
            (n12) -- (n15)
            ;
    \end{tikzpicture}
        
    \caption{Minimal connecting}
    \label{fig:bmlrp-connect}
\end{figure*}


So far we have only discussed which nodes are necessary to connect to. A real implementation will however try to maintain a constant number of closest neighbors on each level. This assumption will be necessary for analysis in sections~\ref{sec:analysis:stretch} and \ref{sec:analysis:overhead}.


%\section{Additional Improvements}
%
%Although, Algorithm~\ref{alg:connect} creates 1 level-$(i+1)$ neighbor for every same-color level-$i$ neighbor and (on average) 3 level-$(i+1)$ neighbors for every level-$i$ neighbor of different color, we expect the higher level neighbors to coincide. In order to balance the routing table size and reduce the path-stretch, Algorithm~\ref{alg:routes} can be modified such that excessive routes are propagated, encouraging robustness and multi-path connectivity.
%
%Another optimization that can reduce the path-stretch is described as follows. For every outcoming data packet with destination $d$, the source node $s$ picks a number $p$ and chooses the next level-$l(s,d)$ hop $a$, prioritizing nodes logically close to $p$. The source node then hashes $p$ $l(s,a)$-number of times and sends the new number $p'$ along the data packet to $a$. The node $a$ (as well as all next hops) similarly selects the next hop $b$, prioritizing nodes logically close to $p'$, and hashes $p'$ corresponding number of times. This way, each number $p$ is mapped to some route. Because a right direction on lower levels is not as significant as routing optimally on higher levels, checking only 100 different values of $p$ should be enough to find an optimal path.
%
%An additional benefit of this method is providing good multi-path connectivity while maximizing the quality of the communication. Suppose, a node has found a number $p$ with a good combination of latency and delivery ratio. If the node increases its traffic load, the quality of the route will naturally go down. By maintaining several different values of $p$ and the associated statistics, the node can balance the load between different paths such that their quality values are kept equal. The node, thus, is able to maximize the quality of its connection with any node for any given amount of traffic load.
%
%Finally, we will examine how nodes behave in a dynamic environment and what optimizations can be employed there. In order to ensure that all known level-$i$ links exist, it is sufficient to send periodic messages between level-0 neighbors only. When some node $a$ goes offline, Algorithm~\ref{alg:routes} guarantees that all level-0 nodes will erase the record of it. All level-1 neighbors of $a$ will be among those level-0 nodes, and will propagate the information to all other level-1 nodes. Continuing the same way, all level-$i$ nodes will remove the node $a$ as well. This strategy, however, creates too much traffic when $a$ constantly goes online and offline. In order to minimize the load, all level-$i$ neighbors of $a$ can keep their $G_{outa}$ variables (see Algorithm~\ref{alg:routes}) until they expire by time or size of the new $G_{outa}$ becomes smaller than the corresponding update.


\section{Analysis}
\label{sec:analysis}

\subsection{Connectivity}
\label{sec:analysis:connectivity}

Theorems~\ref{thm:connectivity-levels} and \ref{thm:connectivity-colors} establish 
two important connectivity properties maintained by the proposed protocol.

\begin{thm}
    Let a level-$i$ network be a connected graph. Then, level-$(i+1)$ network is also a connected graph.

    \begin{proof}
        Assume, level-$(i+1)$ graph is not connected (1). Now, let $c_k$ and $a$ be vertices from two graph components in level-$(i+1)$ graph, such that the distance from $c_k$ to $a$ in level $i$ is minimized (2). Also, without loss of generality assume $c_k$ and $a$ are black in level~$i$. By the assumption (2), vertices $c_1,c_2,...,c_{k-1}$ from the shortest path chain $a$--$c_1$--$c_2$--...--$c_k$ must be white.

        Assume, node $c_k$ does not know about $a$. Then, by Algorithm~\ref{alg:routes} some node $c_j$ does not forward the information about $a$ to $c_{j+1}$ because it knows such black node $b$ that the distance from $b$ to $c_t$ (for some $1 \le t \le j - 1$) is shorter than the distance from $a$ to $c_t$ and the distance from $c_t$ to $c_{j+1}$. If $c_t$ is not connected to either $a$ or $c_k$ in level $(i+1)$, then this must contradict the minimality defined in (2). Otherwise, if $c_t$ is connected to both $a$ and $c_k$ in level $(i+1)$, $a$ and $c_k$ are connected in level $(i+1)$ and this also contradicts (2). Hence, $c_k$ must know a path to $a$. Similarly, $a$ must know a path to $c_k$.

        Algorithm~\ref{alg:connect} ensures that all black neighbors of connected white nodes in level~$i$ are connected in level~$(i+1)$. Therefore, $a$ and $c_k$ must be connected in level~$(i+1)$, leading to the contradiction of the original assumption (1).
    \end{proof}

    \label{thm:connectivity-levels}
\end{thm}

\begin{thm}
    Let a level-$i$ network be a connected graph. Then, every node in this network must know at least one level-$i$ node of the opposite color, if such exists.

    \begin{proof}
        Suppose, a node $c_k$ does not know any node of the opposite color. Without loss of generality, let $c_k$ be black. Now, let $a$ be a white node, such that the distance from $a$ to $c_k$ is minimized. By this assumption, all nodes $c_j$ along the shortest path $a$--$c_1$--$c_2$--...--$c_k$ must be black. Similarly to the previous proof, $c_k$ must know a path to $a$, which contradicts the original assumption.
    \end{proof}

    \label{thm:connectivity-colors}
\end{thm}

Given Theorems~\ref{thm:connectivity-levels}~and~\ref{thm:connectivity-colors}, we are able to say that a node $a$ will always be able to forward a data packet with destination $d$ to the next hop $b$, such that $l(b,d) > l(a,d)$, satisfying the connectivity requirement.


\subsection{Path stretch}
\label{sec:analysis:stretch}

We will now roughly estimate the path-stretch of the proposed protocol. Each data packet is routed on level-0, level-1, ..., level-$k$ networks with respect to the destination until it finally reaches the destination. We assume equal properties across all levels 0, ..., $k - 1$, including each node maintaining a constant number $M$ of closest neighbors on each level.

The path stretch when routing data over level-$(i+1)$ network has to be some constant multiple $q$ ($q \ge 1$) of the stretch when routing data over the underlying level-$i$ network. Because $C_0\sqrt{N}$ is the average distance in a wireless network of $N$ identical uniformly placed nodes~\cite{Kleinrock} and level-$i$ nodes only send data to close level-$(i+1)$ nodes, the distance in physical network between routing level-$i$ node and routing level-$(i+1)$ node must be proportional to $\sqrt{2^i}$. We also know that the maximum level $k$ over which data is routed is less than $log_2N$. We now find the average number of traversed hops:
\begin{equation*}
\begin{split}
    {\rm hops} & = C (1 + \sqrt{2} q + \sqrt{2^2} q^2  + ... + \sqrt{2^{k-1}} q^{k-1} + \sqrt{N} q^k) \\
               & < C \Big( (\sqrt{2} q)^{k-1} \sum_{i = 0}^{\infty} (\sqrt{2}q)^{-i} + \sqrt{N} q^k \Big) \\
               & = C \Big( (\sqrt{2} q)^{k-1} \frac{1}{1 - (\sqrt{2}q)^{-1}} + \sqrt{N} q^k \Big)  \\
               & = C \Big( (\sqrt{2} q)^{k-1} \frac{\sqrt{2} q}{\sqrt{2}q - 1} + \sqrt{N} q^k \Big)  \\
               & < C \Big( \sqrt{N} q^k \frac{\sqrt{2} q}{\sqrt{2}q - 1} + \sqrt{N} q^k \Big)  \\
               & = C \Big( \frac{2\sqrt{2}q - 1}{\sqrt{2}q - 1} \Big) \sqrt{N} q^k  \\
               & < C \Big( \frac{2\sqrt{2}q - 1}{\sqrt{2}q - 1} \Big) \sqrt{N} q^{log_2 N}  \\
               & = C \Big( \frac{2 - \sqrt{2}q}{1 - \sqrt{2}q} \Big) N^{\frac{1}{2} + log_2 q}
\end{split}
\end{equation*}

Thus, the upper bound for the average number of hops is:

\begin{equation*}
    O({\rm hops}) = \Big( \frac{2 - \sqrt{2}q}{1 - \sqrt{2}q} \Big) N^{\frac{1}{2} + log_2 q}
\end{equation*}

Dividing by $\sqrt{N}$ we find the upper bound for the average path stretch:

\begin{equation*}
    O({\rm stretch}) = \Big( \frac{2 - \sqrt{2}q}{1 - \sqrt{2}q} \Big) N^{log_2 q}
\end{equation*}

For example, when each new level increases the path-stretch by 10\%, we have $O({\rm stretch}) \approx \sqrt[7]{N}$ and $O({\rm hops}) \approx N^{0.64}$, which can be considered scalable in many situations. By changing the routing table sizes, $q$ can be adjusted as desired.


\subsection{Protocol overhead}
\label{sec:analysis:overhead}

We will now attempt to estimate size of the protocol overhead under Random Waypoint motion of $N$ uniformly distributed nodes over area $A$ with a constant node density.

Given the smallest level $k$ on which a node $a$ sees all nodes of the level-$k$ network with respect to $a$, the node $a$ must exchange protocol messages with all of its $M$ neighbors on each level $0 \le i \le k$. The average physical distance to the $M$ physical neighbors of $a$ is proportional to $\sqrt{A / N}$. Similarly, the average distance (both geographical and in hops) to $M$ level-$i$ neighbors is proportional to $\sqrt{A / (N / 2^i)} = \sqrt{2^i (A / N)}$. The node density $(A / N)$ is constant; therefore, the average distance to $M$ level-$i$ neighbors is proportional to $\sqrt{2^i}$. Node $a$ connects to a level-$i$ node $b$ and disconnects from the $M$'s level-$i$ neighbor $c$ if the distance to $b$ becomes less than the distance to $c$. Hence, it is reasonable to expect the frequency of connects/disconnects and the frequency of protocol messages exchanged with level-$i$ nodes to be proportional $1 / \sqrt{2^i}$:
%
\begin{equation*}
    F(i) \sim \sqrt{\frac{1}{2^i}}
\end{equation*}

Sending information to a level-$i$ neighbor requires sending the information over level-$(i - 1)$ network. Section~\ref{sec:analysis:overhead} shows that the number of physical hops traversed is bounded by 
%
\begin{equation*}
    1 + \sqrt{2} q + \sqrt{2^2} q^2  + ... + \sqrt{2^{k-1}} q^{k-1} < \frac{ (\sqrt{2}q)^k }{\sqrt{2}q - 1}
\end{equation*}
%
The amount of traffic a single node creates in the network to communicate with its level-$i$ neighbors:
%
\begin{equation*}
    O(T(i)) = \frac{ (\sqrt{2}q)^k }{\sqrt{2}q - 1}
\end{equation*}

Thus, the total protocol overhead a single node creates in the network is given by
%
\begin{equation*}
    O({\rm overhead}) = O\Big( \sum_{i = 0}^{k} F(i)T(i) \Big) = \sum_{i = 0}^{k} \frac{ \big( \frac{q}{\sqrt{2}} \big)^i }{ \sqrt{2}q - 1 }
\end{equation*}
%
For $q < \sqrt{2}$, the protocol overhead is bounded by the logarithm of the total number of nodes $N$.

\subsection{Average node degree}
\label{sec:analysis:nodedegree}
\emph{TODO}


\section{Conclusion}
\label{sec:conclusion}

We have developed a new proactive routing protocol for MANETs, which has several important features in comparison to previous protocols: guaranteed delivery, scalability, absence of single points of failure, low path-stretch, support of multi-path connectivity, {\color{orange}mobility}. The proposed protocol has a similar setup to the \urbanxor\ protocol~\cite{Pasquini}, which employs a Kademlia DHT~\cite{kademlia} inspired approach for building the network structure. Unlike the \urbanxor\ protocol, however, the proposed protocol guarantees that if two nodes are indirectly connected they are able to communicate. Furthermore, the proposed protocol generates {\color{orange}much (no simulation yet)} lower overhead traffic than \urbanxor\ in mobile scenarios.  

Connectivity properties of the proposed protocol were proved, and growth rates of the path stretch and number of hops for routes were given. The average node degree as a function of the level~$i$ network was studied numerically. In ongoing work, we are studying other performance aspects of the protocol through large-scale simulations {\color{orange}using ns3}.


% Generate the bibliography.
\bibliography{bmlrp}
\bibliographystyle{unsrt}

\end{document}
